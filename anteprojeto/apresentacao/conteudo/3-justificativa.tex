\section{Justificativa do projeto}

\begin{frame}{Justificativa}
    Dispensa extração manual de características. \\
    \begin{itemize}
        \item Modelos clássicos como SVM, Random Forest e KNN dependem da extração manual de características. \\
        \item O aprendizado profundo aprende automaticamente representações diretamente dos dados, sendo mais adequado para sinais complexos como vibrações, som e imagens.
    \end{itemize}
\end{frame}

\begin{frame}{Justificativa}
    \begin{itemize}
        \item Trilhos ferroviários apresentam falhas sutis e padrões não lineares.
        \item Redes neurais profundas conseguem capturar essas relações de forma mais precisa do que modelos lineares.
    \end{itemize}
\end{frame}

\begin{frame}{Flexibilidade e desempenho}
    \begin{itemize}
        \item O aprendizado profundo se adapta bem a diferentes tipos de dados (imagens, séries temporais, sensores).
        \item Estudos recentes mostram desempenho superior em tarefas de detecção de anomalias em domínios industriais.
    \end{itemize}
\end{frame}