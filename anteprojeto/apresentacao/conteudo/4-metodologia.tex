\section{Metodologia e plano de trabalho}

\begin{frame}{Metodologia e plano de trabalho}
    Será feito um aprofundamento sobre:
    \begin{itemize}
        \item Técnicas de visão computacional aplicadas à detecção de defeitos em trilhos ferroviários.
        \item Modelos de aprendizado profundo, com foco em:
        \begin{itemize}
            \item Redes neurais;
            \item Vit;
            \item DeiT.
        \end{itemize}
        \item Estratégias de aprendizado por transfêrencia em conjunto de dados reduzidos.
    \end{itemize}
\end{frame}

\begin{frame}{Metodologia e plano de trabalho}
    \begin{itemize}
        \item Coleta e organização dos datasets ferroviários.
        \item Pré-processamento das imagens (normalização, balanceamento, aumento de dados).
        \item Implementação dos modelos de aprendizado profundo (CNN, ViT, DeiT).
        \item Treinamento e validação dos modelos.
        \item Avaliação de desempenho com métricas adequadas (F1, AUC, etc.).
        \item Comparação com resultados da literatura.
    \end{itemize}
\end{frame}

\begin{frame}{Metodologia e plano de trabalho}
    Exemplo de trilho sem pino
    \begin{figure}
        \centering
        \includegraphics[width=0.5\linewidth]{images/pino.jpg}
    \end{figure}
    Fonte: https://www.kaggle.com/datasets/gpiosenka/railway-track-fault-detection-resized-224-x-224/data?select=test
\end{frame}

\begin{frame}{Metodologia e plano de trabalho}
    Exemplo de trilho quebrado
    \begin{figure}
        \centering
        \includegraphics[width=0.5\linewidth]{images/quebrado.jpg}
    \end{figure}
    Fonte: https://www.kaggle.com/datasets/gpiosenka/railway-track-fault-detection-resized-224-x-224/data?select=test
\end{frame} 

\begin{frame}{Metodologia e plano de trabalho}
    Exemplo de trilho sem suporte
    \begin{figure}
        \centering
        \includegraphics[width=0.5\linewidth]{images/sem trilhos.jpg}
    \end{figure}
    Fonte: https://www.kaggle.com/datasets/gpiosenka/railway-track-fault-detection-resized-224-x-224/data?select=test
\end{frame}