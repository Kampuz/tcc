\section{Metodologia e plano de trabalho}

\begin{frame}{Metodologia e plano de trabalho}
    Será feito um aprofundamento sobre:
    \begin{itemize}
        \item Técnicas de visão computacional aplicadas à detecção de defeitos em trilhos ferroviários.
        \item Modelos de aprendizado profundo, com foco em:
        \begin{itemize}
            \item Redes neurais;
            \item Vit;
            \item DeiT.
        \end{itemize}
        \item Estratégias de aprendizado por transfêrencia em conjunto de dados reduzidos.
    \end{itemize}
\end{frame}

\begin{frame}{Metodologia e plano de trabalho}
    Será usado: 
    \begin{itemize}
        \item Datasets de imagens de trilhos ferroviários, contendo imagens com e sem defeitos.
        \item Python e bibliotecas como: Pytorch, Tensorflow, Numpy, entre outras.
    \end{itemize}
\end{frame}

\begin{frame}{Metodologia e plano de trabalho}
    O que será feito?
    \begin{itemize}
        \item Coleta e organização dos datasets ferroviários.
        \item Pré-processamento das imagens (normalização, balanceamento, aumento de dados).
        \item Implementação dos modelos de aprendizado profundo (CNN, ViT, DeiT).
        \item Treinamento e validação dos modelos.
        \item Avaliação de desempenho com métricas adequadas (F1, AUC, etc.).
        \item Comparação com resultados da literatura.
    \end{itemize}
\end{frame}