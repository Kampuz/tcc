\section{Formulação do problema}

\begin{frame}{Formulação do problema}
    \begin{columns}
        \begin{column}{0.5\textwidth}
            Falhas estruturais em trilhos são uma das principais causas de acidentes ferroviários.
        \end{column}

        \begin{column}{0.5\textwidth}
            \begin{figure}
                \centering
                \caption*{Acidentes ferroviários por causa (EUA, 2005-2014)}
                \includegraphics[width=\linewidth]{images/causaAcidentes.png}
                {\footnotesize Fonte: Federal Railroad Administration (BBC), 2015}
            \end{figure}
        \end{column}
    \end{columns}
\end{frame}

\begin{frame}{Formulação do problema}
    \begin{columns}
        \begin{column}{0.5\textwidth}
            Inspeções manuais ainda predominam, o que as torna demoradas e suscetíveis a erros humanos.
        \end{column}

        \begin{column}{0.5\textwidth}
            \begin{figure}
                \centering
                \caption*{Exemplo de manutenção local}
                \includegraphics[width=\linewidth]{images/manutencaoPresencial.png}
                {\footnotesize Fonte: Metodologiaassa, 2020}
            \end{figure}
        \end{column}
    \end{columns}
\end{frame}

\begin{frame}[plain]
    \begin{figure}
        \centering
        % \caption{Ilustração de um neurônio}
        % \includegraphics[width=\linewidth]{images/Blausen_0657_MultipolarNeuron.png}
        % % {\footnotesize Fonte: BruceBlaus, 2013}
        % \label{fig:enter-label}
    \end{figure}
\end{frame}


% \begin{frame}{Formulação do problema}
%     \textcite{alma990001897250206341} explicam as motivações de utilização de redes neurais comparadas com a maneira como o cérebro consegue rapidamente processar informações extremamente complexas (e.g. reconhecer um rosto familiar ou um objeto), enquanto um computador pode demorar muito tempo para tarefas mais simples. Isso ocorre porque o cérebro é constituído por neurônios e por isso que possuem a característica de se adaptar ao ambiente e conseguir aprender com as informações que é exposto. 
% \end{frame}

\begin{frame}{Formulação do problema}
    \begin{figure}
    %     \centering
    %     \caption{Representação de uma rede neural}
    %     \includegraphics[width=0.9\linewidth]{images/NeuralNetwork.png}\\
    %     {\footnotesize Fonte: Pramoditha, 2022}
    %     \label{fig:enter-label}
    \end{figure}
\end{frame}

\begin{frame}[plain]
     \begin{figure}
    %     % \centering
    %     % \caption{Representação de uma rede neural}
    %     \includegraphics[width=1.2\linewidth]{images/NeuralNetwork.png}\\
    %     % {\footnotesize Fonte: Pramoditha, 2022}
    %     % \label{fig:enter-label}
    \end{figure}
\end{frame}

\begin{frame}{Formulação do problema}
    \begin{columns}
        \begin{column}{0.4\textwidth}
            As \textit{CNNs} são uma junção das redes neurais artificiais com camadas que possuem um função de convolução, amostragem e processamento não linear. Esse tipo de \textit{machine learning} também possui a capacidade de extração de características que não são espacialmente dependentes porém que possuem correlação.
        \end{column}
        \begin{column}{0.6\textwidth}
            \begin{figure}
                % \centering
                % \caption{Arquitetura genérica de uma \textit{CNN}}
                % \includegraphics[width=\linewidth]{images/cnn.jpg}
                % {\footnotesize Fonte: Saha, 2018}
                % \label{fig:enter-labea}
            \end{figure}
        \end{column}
    \end{columns}
    \nocite{8308186,Goodfellow-et-al-2016,zhang2023dive,LeCun2015}
\end{frame}

\begin{frame}[plain]
    \begin{figure}
        % \centering
        % % \caption{Arquitetura genérica de uma \textit{CNN}}
        % \includegraphics[width=\linewidth]{images/cnn.jpg}
        % % {\footnotesize Fonte: Saha, 2018}
        % % \label{fig:enter-labea}
        \end{figure}
\end{frame}
% \begin{frame}{Formulação do problema}
%     Como as \textit{CNN} são construídas com diversas camadas, cada camada possuí uma função e pesos para que no final consiga abstrair e classificar apropriadamente. Devido a isso, elas são as mais robustas para tarefas de classificação, segmentação, detecção e outros tipos de atividades relacionadas \cites{8308186}{Goodfellow-et-al-2016}{zhang2023dive}{LeCun2015}.
% \end{frame}

% \begin{frame}{Formulação do problema}
%     Durante o treino de uma \textit{CNN} é possível obter algumas métricas importantes como a precisão e a perda para identificar o quanto uma \textit{CNN} está ``aprendendo'' e o quanto está errando na classificação. Uma limitação que as \textit{CNNs} possuem é a necessidade de uma grande quantidade de dados para atingir uma precisão minimamente satisfatória na etapa de treinamento.
%     Durante o treino, possuímos essas métricas:
%     \begin{itemize}
%         \item 
%     \end{itemize}
% \end{frame}

\begin{frame}{Formulação do problema}
    % Geralmente, nem sempre é possível obter uma quantidade grande de dados que seja suficiente para o treino e sua generalização para o para dados do ``mundo real'', isto é, dados que não são sintéticos. Algumas técnicas podem ser empregadas para tentar amenizar essa limitação, por exemplo a aplicação do método de aumento de dados, ou \textit{data augmentation}.
    \begin{itemize}
        \item Geralmente nem sempre é possível obter uma grande quantidade de imagens para o \textbf{treino}
        \item As imagens que vão ser classificadas pelas \textit{CNN} já \textbf{treinada} nem sempre estão em um ``bom estado''
        \begin{itemize}
            \item O que pode acarretar em uma classificação errada se a \textit{CNN} não for robusta
        \end{itemize}
    \end{itemize}
\end{frame}

\begin{frame}{Formulação do problema}
    % Também devemos incluir um fato que nem sempre conseguimos obter imagens nítidas ou ideais para sua classificação. Para avaliar o comportamento do desempenho em imagens pré-processadas, adicionaremos ruídos ou filtros (dentro do que faz sentido para o domínio do problema) gradativamente e avaliar nas arquiteturas de \textit{CNNs} treinados com dados sem pré-processamento.
    Nem sempre é possível obter imagens ``boas'' para a análise, podemos ter problemas durante a captura da fotografia:
    \begin{itemize}
        \item Brilho
        \item Contraste
        \item Movimentação, tornando a imagem borrada
        \item e entre outros
    \end{itemize}
\end{frame}