\section{Metodologia e plano de trabalho}
    Inicialmente, será conduzido um estudo teórico aprofundado sobre o tema, abrangendo métodos de aprendizado profundo, com ênfase em Máquina de Vetores de Suporte (SVM), dado o maior volume de pesquisas na área, e em Redes Neurais, reconhecida como uma das mais promissoras, mais ainda pouco explorada no contexto ferroviário. Além disso, serão estudadas técnicas de análise de dados e o funcionamento das linhas férreas brasileiras.
    
    Na etapa seguinte, serão realizadas a coleta e organização dos dados do setor ferroviário, com possível adequação ao padrão \textit{General Transit Feed Specification} (GTFS, Especificação Geral de Feed de Transporte Público), caso necessário. \cite{SARHANI2024100120}
    
    Em seguida, serão aplicados os métodos relevantes sobre os dados coletados. Por fim, será elaborado um artigo científico com o objetivo de relatar os resultados obtidos e discutir as contribuições do trabalho.