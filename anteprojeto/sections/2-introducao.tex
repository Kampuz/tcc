\setstretch{1.5}
\section{Formulação do problema}
    No Brasil, o setor ferroviário desempenha um papel estratégico no transporte de cargas e vem sendo cada vez mais incentivado a expandir sua atuação para o transporte de passageiros \cite{MT_2025_ferroviario}, sendo considerado essencial para a economia brasileira e apresentando grande potencial como um método de transporte público interestadual.
    
    Entretanto, as operações ferroviárias enfrentam desafios recorrentes, como atrasos, falhas em equipamentos, degradação de infraestrutura e ineficiência no planejamento de rotas e horários. Esses fatores comprometem a eficiência e confiabilidade do sistema, elevando custos operacionais e comprometendo a adoção do público ao transporte ferroviário (Tiong, K. Y. 2024).

    Nesse contexto, técnicas de inteligência artificial e ciência de dados têm se mostrado promissoras para melhorar a eficiência ferroviária. Estudos internacionais demonstram que algoritmos de aprendizado de máquina, como Máquinas de  Vetor de Suporte (SVMs, Support Vector Machines) e redes neurais, já vêm sendo aplicadas com sucesso na previsão de atrasos e acidentes, além de auxiliar no planejamento de rotas \cite{https://doi.org/10.1155/2023/1832501, SARHANI2024100120, 10468590}.

    Diante desse cenário, este trabalho tem como objetivo investigar a aplicação de técnicas de inteligência artificial e ciência de dados na previsão e otimização de processos ferroviários brasileiros, buscando melhorar a eficiência operacional, antecipar atrasos, otimizar escalas e facilitar a manutenibilidade do sistema. Dessa forma, pretende-se também ampliar a literatura da área, com ênfase no uso de redes neurais - técnicas reconhecidas pelo seu potencial, mas ainda pouco exploradas na literatura \cite{SARHANI2024100120, TIONG2023104027}.