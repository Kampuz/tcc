% Define o tipo de documento como "article", com fonte de 12pt
% A opção 'nodisplayskipstretch' evita que o LaTeX altere o espaçamento ao redor de fórmulas exibidas
\documentclass[12pt,nodisplayskipstretch]{article}

% Importa o arquivo com as configurações gerais do projeto (pacotes, margens, fonte etc.)
\usepackage{config/commands}

% Importa o arquivo de configurações adicionais (talvez pacotes específicos ou definições de estilo)
\usepackage{config/config}

\begin{document} % Início do conteúdo do documento
% ----------------------------
% SEÇÕES DO DOCUMENTO
% ----------------------------

% Importa a capa do trabalho
\input{sections/0-capa}
\newpage
% Importa a seção de identificação do projeto (pode conter título, autor, orientador, etc.)
\setstretch{1.5}
\section{Identificação}

\subsection{Título do projeto}
    \titulo

\subsection{Participantes}
Orientador: \orientador
Aluno: \nome

\subsection{Área ou linha de pesquisa}
Grande área: Ciências Exatas e da Terra.

Área: Ciência da Computação

Subárea: Ciência de Dados

Especialidade: Inteligência Artificial

\subsection{Palavras Chaves}
Transporte Ferroviário, Otimização de Operações, Inteligência Artificial, Reconhecimento de Padrões.

\subsection{Duração}
\begin{table}[H]
    \begin{tabular}{ll}
        Início: & Agosto/2025\\
        Término: & Novembro/2026\\
    \end{tabular}
\end{table}
\newpage
% Importa a introdução do projeto, onde o tema é apresentado
\setstretch{1.5}
\section{Formulação do problema}
    No Brasil, o setor ferroviário desempenha um papel estratégico no transporte de cargas e vem sendo cada vez mais incentivado a expandir sua atuação para o transporte de passageiros \cite{MT_2025_ferroviario}, sendo considerado essencial para a economia brasileira e apresentando grande potencial como um método de transporte público interestadual.
    
    Entretanto, as operações ferroviárias enfrentam desafios recorrentes, como atrasos, falhas em equipamentos, degradação de infraestrutura e ineficiência no planejamento de rotas e horários. Esses fatores comprometem a eficiência e confiabilidade do sistema, elevando custos operacionais e comprometendo a adoção do público ao transporte ferroviário (Tiong, K. Y. 2024).

    Nesse contexto, técnicas de inteligência artificial e ciência de dados têm se mostrado promissoras para melhorar a eficiência ferroviária. Estudos internacionais demonstram que algoritmos de aprendizado de máquina, como Máquinas de  Vetor de Suporte (SVMs, Support Vector Machines) e redes neurais, já vêm sendo aplicadas com sucesso na previsão de atrasos e acidentes, além de auxiliar no planejamento de rotas \cite{https://doi.org/10.1155/2023/1832501, SARHANI2024100120, 10468590}.

    Diante desse cenário, este trabalho tem como objetivo investigar a aplicação de técnicas de inteligência artificial e ciência de dados na previsão e otimização de processos ferroviários brasileiros, buscando melhorar a eficiência operacional, antecipar atrasos, otimizar escalas e facilitar a manutenibilidade do sistema. Dessa forma, pretende-se também ampliar a literatura da área, com ênfase no uso de redes neurais - técnicas reconhecidas pelo seu potencial, mas ainda pouco exploradas na literatura \cite{SARHANI2024100120, TIONG2023104027}.
\newpage
% Importa a seção de objetivos do projeto, onde são definidos os objetivos gerais e específicos
\section{Objetivos do projeto}
\subsection{Objetivos gerais}
    Este projeto tem como objetivo realizar estudos sobre o uso de inteligência artificial e ciência de dados aplicadas ao setor ferroviário, com foco em previsão e otimização de operações.

\subsection{Objetivos específicos}

\begin{itemize}
    \item 1. Estudar e aplicar conceitos relacionados à inteligência artificial e ciência de dados, estudados na graduação, assim como previsão em sistemas de transporte.
    \item 2. Realizar levantamento e análise de dados do setor ferroviário brasileiro.
    \item 3. Aplicar métodos de aprendizado de máquina para previsão de atrasos, falhas e demandas operacionais.
    \item 4. Comparar os resultados obtidos com trabalhos similares na literatura, identificando vantagens e limitações.
\end{itemize}
% Importa a seção de metodologia, onde são descritos os métodos que serão utilizados no projeto
\section{Metodologia e plano de trabalho}
    Inicialmente, será conduzido um estudo teórico aprofundado sobre o tema, abrangendo métodos de aprendizado profundo, com ênfase em Máquina de Vetores de Suporte (SVM), dado o maior volume de pesquisas na área, e em Redes Neurais, reconhecida como uma das mais promissoras, mais ainda pouco explorada no contexto ferroviário. Além disso, serão estudadas técnicas de análise de dados e o funcionamento das linhas férreas brasileiras.
    
    Na etapa seguinte, serão realizadas a coleta e organização dos dados do setor ferroviário, com possível adequação ao padrão \textit{General Transit Feed Specification} (GTFS, Especificação Geral de Feed de Transporte Público), caso necessário. \cite{SARHANI2024100120}
    
    Em seguida, serão aplicados os métodos relevantes sobre os dados coletados. Por fim, será elaborado um artigo científico com o objetivo de relatar os resultados obtidos e discutir as contribuições do trabalho.

% Importa a seção de equipamento e material necessário para o desenvolvimento do projeto
\section{Equipamento e material}
    
\begin{itemize}
    \item Artigos;
    \item Internet;
    \item Computador Desktop;
    \item Dados Ferroviários;
\end{itemize}

% Importa o cronograma de execução do projeto, com as etapas e os prazos
\section{Cronograma de execução}
As atividades a serem executadas estão listadas a seguir:
\begin{enumerate}
    \item Elaboração, entrega e apresentação do anteprojeto
    \item Elaboração e entrega da revisão bibliográfica
    \item Realização de testes 
    \item Escrita do artigo científico
    \item Revisão e apresentação do artigo
\end{enumerate}

O cronograma será dividido em 9 bimestres:

\begin{table}[H]
    \centering
    \begin{tabular}{|c|c|c|c|c|c|c|c|c|c|c|c|c|}
    \hline
        Período & 1º & 2º & 3º & 4º & 5º & 6º & 7º & 8º & 9º\\
    \hline
        1 & • &  &  &  & &  &  &  & \\
    \hline
        2 & • & • &  &  &  &  &  &  &  \\
    \hline
        3 &  & • & • & • & •  &  &  &  &  \\
    \hline
        4 &  &  &  & • & • & • & • &  &  \\
    \hline
        5 &  &  &  &  &  &   & • & • & • \\
    \hline
    \end{tabular}
\end{table}


% Importa as referências bibliográficas utilizadas no projeto
\newpage
\section{Referências}
Tiong, K. Y. (2024). Data-driven Train Delay Prediction. [Doctoral Thesis (compilation), Faculty of Engineering,
LTH]. Lund University Faculty of Engineering, Technology and Society, Transport and Roads, Lund, Sweden.

Marques, L., Moro, S. \& Ramos, P. Data-driven insights to reduce uncertainty from disruptive events in passenger railways. Public Transp (2025). https://doi.org/10.1007/s12469-024-00380-9
\nocite{*}
\printbibliography

\end{document} % Fim do conteúdo do documento
