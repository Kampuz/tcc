\section{Introdução}

O setor ferroviário possui papel estratégico no transporte de cargas e passageiros em escalas continentais. Estudos recentes destacam o potencial do modal ferroviário para aumenter a eficiência logística e reduzir custos operacionais, além de contribuir para a mobilidade sustentável.
Entretanto, a infraestrutura ferroviária enfrenta problemas recorrentes, como desgaste de trilhos, falhas mecânicasa, degradação de componentes estruturais e atrasos operacionais, os quais comprometem a segurança e a confiabilidade do sistema \cite{TIONG2023104027}.

Tradicionalmente, inspeções ferroviárias são realizadas por equipes especializadas, de for\-ma manual e periódica. Embora essenciais, tais procedimentos tendem a ser lento, subjetivos e onerosos, podendo representar mais de 50\% dos custos totais de manutenção em alguns países \cite{HUANG201846}. Além do mais, a depêndencia exclusiva da inspeção humana dificulta a detecção precoce de falhas, especialmente em redes extensas ou de díficil acesso.

Com o avanço da automação e das tecnologias associadas à Indústria 4.0, métodos de inspeção não destrutiva (NDE) têm ganhado destaque. Dispositivos embarcados, sensores, sistemas ópticos, varredura a laser e imageamento de alta resolução permitem monitoramento contínuo e preciso da infraestrutura sem interromper a operação ferroviária \cite{TIONG2023104027}.
Esses avanços tornaram viável o desenvolvimento de sistemas inteligentes capazes de identificar anomalias com maior agilidade e confiabilidade.

No entanto, métodos clássicos de processamento de imagens apresentam limitações diante da elevada variabilidade visual do ambiente ferroviário e das taxas significantemente altas de falsos positivos. \cite{SARHANI2024100120}. Esse cenário evidencia a necessidade de técnicas mais robustas e adaptáveis, capazes de lidar com a complexidade e diversidade das imagens coletadas em campo.

\subsection{Justificativa}

Diante desses avanços, este trabalho concentra-se na investigação de métodos de aprendizado profundo aplicados à detecção automática de anomalias em trilhos ferroviários a partir de imagens.
O objetivo é avaliar diferentes modelos, analisar seu desempenho e discutir suas potencialidades como ferramentas de apoio à manutenção preventiva da infraestrutura ferroviária.
Por meio da revisão, sistematização e analíse crítica da literatura, busca-se identificar tendências, limitações e oportunidades de pesquisa, contribuindo para o desenvolvimento de soluções mais eficientes e acessíveis para o setor.

\subsection{Objetivos}

Sendo assim, este trabalho tem como objetivo investigar o uso de técnicas de aprendizado profundo para a detecção de anomalias em dados ferroviários, buscando desenvolver e avaliar modelos capazes de identificar automaticamente falhas estruturais em trilhos por meio de imagens.
Para isso, serão conduzidas etapas de pré-processamento e análise exploratória dos dados, implementação de diferentes arquiteturas de aprendizado profundo, avaliação de desempenho por meio de métricas apropriadas e comparação dos resultados com estudos relacionados da literatura.