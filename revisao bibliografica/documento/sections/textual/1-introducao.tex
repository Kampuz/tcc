\section{Introdução}

O setor ferroviário possui papel estratégico no transporte de cargas e passageiros. Estudos recentes destacam o potencial do modal ferroviário para aumenter a eficiência logística e reduzir custos operacionais, além de contribuir para a mobilidade sustentável \cite{MT_2025_ferroviario}
Entretanto, a infraestrutura ferroviária enfrenta problemas recorrentes, como desgaste de trilhos, falhas mecânicasa, degradação de componentes estruturais e atrasos operacionais, os quais compormentem a segurança e a confiabilidade do sistema \cite{TIONG2023104027}.

Com o avanço das técnicas de inteligência artificial, diversos trabalhos têm explorado o uso de métodos de aprendizado de máquina e aprendizado profundo para monitoramento
%\subsection{Justificativa}

%\subsection{Objetivos}