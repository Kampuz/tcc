\section{Introdução}

% O setor ferroviário possui papel estratégico no transporte de cargas e passageiros. Estudos recentes destacam o potencial do modal ferroviário para aumenter a eficiência logística e reduzir custos operacionais, além de contribuir para a mobilidade sustentável \cite{MT_2025_ferroviario}
% Entretanto, a infraestrutura ferroviária enfrenta problemas recorrentes, como desgaste de trilhos, falhas mecânicasa, degradação de componentes estruturais e atrasos operacionais, os quais compormentem a segurança e a confiabilidade do sistema \cite{TIONG2023104027}.

% Tradicionalmente, a inspeção ferroviária é realizada por equipes especializadas, de forma manual e periódica. Embora essencial, esse processo é lento, subjetivo e custoso, chegando a representar mais de 50\% dos custos totais de manutenção em alguns países \cite{HUANG201846}.
% Além disso, a depêndencia exclusiva da inspeção humana dificulta a detecção precoce de falhas em redes extensas ou de díficil acesso.

% Com o avanço da automação e das tecnologias associadas à Indústria 4.0, métodos de inspeção não destrutiva (NDE) têm ganhado destaque. Dispositivos embarcados, sensores, sistemas ópticos, varredura a laser e imageamento de alta resolução permitem monitoramento contínuo e preciso da infraestrutura sem interromper a operação ferroviária (Tiong; Ma; Palmqvist, 2023). Esses avanços possibilitam o desenvolvimento de sistemas inteligentes capazes de identificar anomalias com maior agilidade e confiabilidade.

% Apesar disso, métodos clássicos de processamento de imagens apresentam limitações diante da elevada variabilidade visual do ambiente ferroviário, levando a taxas significativas de falsos positivos. Além disso, modelos de aprendizado profundo treinados do zero exigem grandes quantidades de dados rotulados, recurso escasso quando se trata de defeitos reais, que são eventos raros (Sarhani; Voß, 2024).

\subsection{Justificativa}

Diante desses avanços, este trabalho concentra-se na investigação de métodos de aprendizado profundo aplicados à detecção automática de anomalias em trilhos ferroviários a partir de imagens.
O objetivo é avaliar diferentes modelos, analisar seu desempenho e discutir suas potencialidades como ferramentas de apoio à manutenção preventiva da infraestrutura ferroviária.
Ao reunir e analisar a literatura existente, busca-se destacar tendências, limitações e oportunidades de pesquisa, contribuindo para o desenvolvimento de soluções mais eficientes e acessíveis para o setor.

\subsection{Objetivos}

Sendo assim, este trabalho tem como objetivo investigar o uso de técnicas de aprendizado profundo para a detecção de anomalias em dados ferroviários, buscando desenvolver e avaliar modelos capazes de identificar automaticamente falhas estruturais em trilhos por meio de imagens.
Para isso, serão conduzidas etapas de pré-processamento e análise exploratória dos dados, implementação de diferentes arquiteturas de aprendizado profundo, avaliação de desempenho por meio de métricas apropriadas e comparação dos resultados com estudos relacionados da literatura.