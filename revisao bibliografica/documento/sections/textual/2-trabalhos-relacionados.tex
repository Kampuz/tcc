%5-6 paginas

\section{Trabalhos Relacionados}



\begin{table}[H]
\centering
\caption{Resumo de trabalhos relacionados em detecção de defeitos ferroviários}
\begin{tabular}{|p{3cm}|p{3cm}|p{3cm}|p{3cm}|}
\hline
\textbf{Autor / Ano} & \textbf{Modelo} & \textbf{Tipo de dado} & \textbf{Resultados / Observações} \\
\hline
Karakose et al., 2017 & Métodos de visão clássica & Vídeo e imagens de trilhos & Limitado por falta de dados rotulados \\
\hline
Szegedy et al., 2015 & Inception CNN & Imagens gerais & Transfer learning para datasets pequenos \\
\hline
Dai et al., 2019 & AlexNet, ResNet & Trilhos / fixadores & Boa generalização usando pré-treinamento \\
\hline
Rail-Former, 2023 & Transformer semântico & Imagens de superfície de trilhos & Alta mIoU, captura detalhes pequenos \\
\hline
TrackNet, 2022 & Vision Transformer & Trilhos de alta velocidade & Melhoria em precisão e F1, usa transfer learning \\
\hline
MobileViT, 2022 & Compact Transformer & Imagens limitadas & Modelo leve, adequado para tempo real \\
\hline
EdgeViT, 2022 & Compact Transformer & Imagens limitadas & Eficiência computacional em CPU ARM \\
\hline
CTBM-DAHD, 2023 & CNN + Transformer & Arcing horns & Maior recall e precisão em alta velocidade \\
\hline
\end{tabular}
\label{tab:trabalhos-relacionados}
\end{table}

\subsection{Relação com o meu trabalho}
