\section{Metodologia}

\subsection{Datasets}


Um dos datasets utilizados neste trabalho será \textit{Railway Track fault Detection Resized (224 X 224)}, disponibilizado na plataforma Kaggle por \textcite{gpiosenka_railway_2022}.
Trata-se de uma versão redimensionada do conjunto original \textit{Railway Track Fault Detection} disponibilizado por \textcite{hossain2021railway}, também na plataforma Kaggle.

As imagens originais possuem alta resolução, o que torna o pré-processamento custoso, por esse motivo o dataset foi recriado com todas as imagens reduzidas para 244x244x3, facilitando seu uso, também foi incluido um arquivo \textit{rails.csv}, que simplifica o carregamento e a organização das amostras \cite{gpiosenka_railway_2022}.

Também será utilizado o dataset \textit{Railway Track Fault Detection | Dataset2(Fastener)}, disponibilizado na plataforma Kaggle por \textcite{ashikadnan_railway_2021}.
Porém, ao contrário do conjunto anterior, esse conjunto contém apenas imagens de pinos de fixação e não foi redimensionado.

\subsubsection{informações sobre os datasets}

Ambos os datasets são compostos pelo mesmo número de imagens com e sem anomalias.

O conjunto de \textcite{gpiosenka_railway_2022} contém 384 imagens de trilhos divididas entre 300 imagens de treinamento, 22 imagens de teste e 62 imagens de validação. Esse conjunto contém tres tipos de anomalias:  ausência de pinos de fixação, deformações e rachaduras nos trilho e danos na fundação do trilho (parte de madeira transversal ao trilho).

Enquanto que o conjunto de \textcite{ashikadnan_railway_2021} contém 1399 imagens de pinos de fixação divididas entre 980 imagens de treinamento, 140 imagens de teste 280 imagens de validação

\begin{figure}[H]
    \caption{Conjunto de imagens do dataset em grade 2x2}
    \centering
    \begin{subfigure}[b]{0.45\linewidth}
        \centering
        \includegraphics[width=0.8\linewidth]{images/dataset/boas condicoes.jpg}
        \caption{Sem defeito}
        \label{fig:img1}
    \end{subfigure}
    \hfill
    \begin{subfigure}[b]{0.45\linewidth}
        \centering
        \includegraphics[width=0.8\linewidth]{images/dataset/falta pino.jpg}
        \caption{Defeituoso (ausência de pino superior)}
        \label{fig:img2}
    \end{subfigure}
    
    \vspace{0.25cm}
    
    \begin{subfigure}[b]{0.45\linewidth}
        \centering
        \includegraphics[width=0.8\linewidth]{images/dataset/fundacao.jpeg}
        \caption{Defeituoso (fundação)}
        \label{fig:img3}
    \end{subfigure}
    \hfill
    \begin{subfigure}[b]{0.45\linewidth}
        \centering
        \includegraphics[width=0.8\linewidth]{images/dataset/rachadura.jpg}
        \caption{Defeituoso (deformação nos trilhos)}
        \label{fig:img4}
    \end{subfigure}
    
    \fonteimagem{Fonte: \textcite{gpiosenka_railway_2022}}
    \label{fig:grade4}
\end{figure}

Como mencionado anteriormente, o segundo dataset não foi previamente redimensionado. Portanto suas imagens serão ajustadas para o formato 224x244x3,a fim de padronizar as dimensões e minimizar a heterogeneidade entre os diferentes conjuntos de dados.

\begin{figure}[H]
    \caption{Conjunto de imagens do segundo dataset}
    \centering
    \begin{subfigure}[b]{0.45\linewidth}
        \centering
        \includegraphics[width=0.8\linewidth]{images/dataset/2defeituoso.jpg}
        \caption{Defeituoso}
        \label{fig:img1 dataset2}
    \end{subfigure}
    \hfill
    \begin{subfigure}[b]{0.45\linewidth}
        \centering
        \includegraphics[width=0.8\linewidth]{images/dataset/2ok.jpg}
        \caption{Sem defeito}
        \label{fig:img2 dataset2}
    \end{subfigure}
\end{figure}

\subsection{Modelos}

\subsubsection{CNN pré-treinada}

Será utilizada uma redes neural convolucionais pré-treinada para aproveitar o conhecimento adquirido por grandes modelos treinados em bases extensas como ImageNet e minimizar os impactos causados pelo uso de um dataset reduzido.

A arquitetura escolhida foi a MobileNetV3, sendo esta uma melhoria do mobileNetV2 utilizada por \textcite{MobileNetV3} em seu trabalho de identificação de defeitos em pinos de fixação e amplamente recomendada para tratar de datasets pequenos, como é o caso.

\subsubsection{SVM}

Neste modelo híbrido, uma CNN pré-treinada será utilizada como extratora de características. As últimas camadas convolucionais produzem vetores representativos do conteúdo visual da imagem, que são utilizados como entrada para a SVM.


\subsubsection{ViT e DeiT}


\subsection{Resultados Esperados}

Espera-se que os modelos desenvolvidos sejam capazes de identificar com elevada precisão os diferentes tipos de defeitos presentes nas imagens do conjunto de dados.
Em particular, os resultados esperados incluem:

\begin{itemize}
\item Os modelos baseados em aprendizado profundo obtenham desempenho superior aos métodos clássicos.
\item As arquiteturas que utilizam transferência de aprendizado apresentem resultados mais estáveis e eficientes mesmo com um conjunto de dados limitado.
\item Os modelos sejam capazes de generalizar para imagens de contextos distintos, mantendo desempenho consistente mesmo diante de variações de iluminação, ângulo ou ruído.
\item A análise comparativa permita identificar quais modelos são mais adequados para detecção automática de defeitos ferroviários e justificar suas vantagens técnicas.
\end{itemize}

\subsection{Avaliação dos Resultados}

Os resultados dos testes serão obtidos através de uma matriz de confusão, que organiza as predições do modelo em quatro categorias fundaamentais: verdadeiros positivos (TP) falsos positivos (FP), verdadeiros negativos (TN) e falsos negativos (FN).
Essa estrutura permite avaliar não apenas o desempenho geral do classificador, mas também entender em quais situações ele tende a errar.

\begin{figure}[H]
    \centering
    \caption{Imagem mostrando o exemplo de uma matriz de confusão}
    \includegraphics[width=0.5\linewidth]{images/matriz de confusão.png}\\
    \fonteimagem{Fonte: \textcite{brains2023performance}}
    \label{fig:matriz de confusão}
\end{figure}

A partir da matriz de confusão, quatro métricas amplamente utilizadas na avaliação de modelos de classificação serão calculadas:

\begin{itemize}
    \item \textbf{Acurácia} (Accuracy) \\
    Mede a proporção de classificações corretas em relação ao total de predições realizadas, sendo utilizada para analisar o desempenho dos modelos de forma geral.
    \[
        A = \frac{TP + TN}{TP + FP + FN +TN}
    \]

    \item \textbf{Precisão} (Precision)\\
    Mede a proporção de positivos verdadeiros entre todas as predições positivas, sendo utilizada para avaliar o grau de confiabilidade das detecções positivas.
    \[
        P = \frac{TP}{TP + FP}
    \]

    \item \textbf{Revocação} (Recall) \\
    Mede a capacidade do modelo de recuperar os positivos reais, avaliando o quanto o modelo deixa de detectar casos positivos.
    \[
        R = \frac{TP}{TP + FN}
    \]
    
    \item \textbf{F1-score} \\
    Média harmônica entre precisão e revocação.
    Resume ambas as métricas em um único valor equilibrado.
    \[
        F1 = 2 \cdot \frac{P \cdot R}{P + R}
    \]
\end{itemize}

Após a obtenção dos resultados, será realizada uma comparação com os estudos presentes na literatura, analisando como os diferentes modelos se comportaram no contexto deste trabalho em relação ao desempenho observado em suas aplicações originais.
Essa análise permitirá identificar em que medida fatores como domínio domínio dos dados, condições de captura das imagens e complexisdade das anomalias influenciam o desempenho das arquiteturas, além de evidenciar quais modelos apresentam maior capacidade de generalização para o cenário de detecção de falhas em trilhos ferroviários.