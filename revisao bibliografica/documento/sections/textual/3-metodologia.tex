\section{Metodologia}

\subsection{Dataset}

O dataset inicial que será utilizado neste trabalho é o \textit{Railway Track fault Detection Resized (224 X 224)}, disponibilizado na plataforma Kaggle por Gerry.
Trata-se de uma versão redimensionada do conjunto original \textit{Railway Track Fault Detection}.
Segundo o autor, as imagens possuem alta resolução, o que torna o pré-processamento custoso; por esse motivo o dataset foi recriado com todas as imagens reduzidas para 244x244x3, facilitando seu uso, também foi incluido um arquivo \textit{rails.csv}, que simplifica o carregamento e a organização das amostras.

O conjunto contém 384 imagens de trilhos, sendo metade correspondente a trilhos defeituosos e a outra metade a trilhos em boas condições.
A divisão proposta pelo autor distribui as imagens em 6\% para teste, 78\% para treinamento e 16\% para validação. 

Os defeitos representados no dataset concentram-se em três categorias: ausência de pinos de fixação, deformações na alma do trilho (parte lateral entre a)

Ele é composto por três tipos de defeitos, falta de pinos, deformações/rachaduras nos trilhos e ausência de dormentes (tábuas transversais aos trilhos)

\begin{figure}[H]
    \centering
    % Primeira linha
    \begin{subfigure}[b]{0.45\linewidth}
        \centering
        \includegraphics[width=1\linewidth]{images/dataset/boas condicoes.jpg}
        \caption{Imagem sem defeito}
        \label{fig:img1}
    \end{subfigure}
    \hfill
    \begin{subfigure}[b]{0.45\linewidth}
        \centering
        \includegraphics[width=1\linewidth]{images/dataset/falta pino.jpg}
        \caption{pino superior faltando}
        \label{fig:img2}
    \end{subfigure}

    \vspace{0.5cm} % Espaço entre linhas

    % Segunda linha
    \begin{subfigure}[b]{0.45\linewidth}
        \centering
        \includegraphics[width=1\linewidth]{images/dataset/fundacao.jpeg}
        \caption{defeito na fundação}
        \label{fig:img3}
    \end{subfigure}
    \hfill
    \begin{subfigure}[b]{0.45\linewidth}
        \centering
        \includegraphics[width=1\linewidth]{images/dataset/rachadura.jpg}
        \caption{deformação/rachadura nos trilhos}
        \label{fig:img4}
    \end{subfigure}

    \caption{Conjunto de imagens em grade 2x2. Fonte: Gerry (2022)}
    \label{fig:grade4}
\end{figure}

Devido ao tamanho reduzido e à qualidade limitada do dataset, permanece em aberto a possibilidade de incorporar novos conjuntos de dados, seja para ampliar a quantiade de amostras disponíveis, seja para garantir maior diversidade visual e melhorar o desempenho dos modelos.

\subsection{Modelos}

\subsection{Resultados Esperados}

\subsection{Avaliação dos Resultados}

Os resultados dos testes serão obtidos através de uma matriz de confusão com o objetivo de determinar

\begin{figure}[H]
    \centering
    \caption{Imagem mostrando o exemplo de uma matriz de confusão}
    \includegraphics[width=0.5\linewidth]{images/matriz de confusão.png}\\
    \fonteimagem{Fonte: \textcite{brains2023performance}}
    \label{fig:matriz de confusão}
\end{figure}

Os resultados dos modelos serão avaliados utilizando quatro métricas que são amplamente utilizadas na avaliação de modelos de classificação:

\begin{itemize}
    \item \textbf{Acurácia} (Accuracy) \\
    Mede a proporção de classificações corretas em relação ao total de predições realizadas, sendo utilizada para analisar o desempenho dos modelos de forma geral.
    \[
        A = \frac{CC}{CC + CI}
    \]

    \item \textbf{Precisão} (Precision)\\
    Mede a proporção de positivos verdadeiros entre todas as predições positivas, sendo utilizada para avaliar o grau de confiabilidade das detecções positivas.
    \[
        P = \frac{VP}{VP + FP}
    \]

    \item \textbf{Revocação} (Recall) \\
    Mede a capacidade do modelo de recuperar os positivos reais, avaliando o quanto o modelo deixa de detectar casos positivos.
    \[
        R = \frac{VP}{VP + FN}
    \]
    
    \item \textbf{F1-score} \\
    Média harmônica entre precisão e revocação, resumindo ambas em uma única métrica.
    \[
        F1 = 2 \cdot \frac{P \cdot R}{P + R}
    \]
\end{itemize}

\noindent\textbf{Legenda das siglas:}
\begin{itemize}
    \item CC: Classificações Corretas;
    \item CI: Classificações Incorretas;
    \item VP: Verdadeiros Positivos;
    \item FP: Falsos Positivos;
    \item FN: Falsos Negativos;
\end{itemize}

Após a obtenção dos resultados, será feito uma comparação com os estudos da literatura, buscando descobrir o motivo