% Resumo/palavras-chave (língua vernácula)------------------------------------------------------
\newpage
\thispagestyle{empty}
\begin{center}
    \uppercase{\textbf{Resumo}}
\end{center}
A detecção automática de defeitos ferroviários tem se tornado um tema de grande relevância devido à necessidade de aumentar a segurança, reduzir custos de manutenção e melhorar a eficiência operacional das ferrovias. Métodos tradicionais de inspeção, realizados manualmente, são caros, lentos e sujeitos a erros humanos, o que motiva o uso de técnicas de visão computacional e aprendizado profundo. Este trabalho apresenta uma análise comparativa de diferentes modelos de deep learning aplicados à detecção de anomalias em trilhos ferroviários por meio de imagens. São investigadas arquiteturas clássicas de redes convolucionais, métodos híbridos e transformadores visuais, além de modelos especializados projetados para o domínio ferroviário. O estudo inclui etapas de pré-processamento e transferência de aprendizado, seguindo as melhores práticas encontradas na literatura. Os resultados esperados envolvem melhor desempenho em relação aos métodos tradicionais, maior robustez em condições variadas e identificação das arquiteturas mais adequadas para a tarefa. O objetivo geral é contribuir para o desenvolvimento de tecnologias mais precisas e acessíveis para manutenção preditiva da infraestrutura ferroviária.


\begin{center}
    \uppercase{\textbf{Palavras-chaves}}
\end{center}
Detecção de anomalias; Visão computacional; Aprendizado profundo; Transferência de aprendizado; Transformadores visuais; Infraestrutura ferroviária.
