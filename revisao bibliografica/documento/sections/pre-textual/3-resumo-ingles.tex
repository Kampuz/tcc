\newpage
\thispagestyle{empty}
\begin{center}
    \textbf{ABSTRACT}
\end{center}
The automatic detection of railway defects has become increasingly relevant due to the growing demand for improving safety, reducing maintenance costs, and enhancing the operational efficiency of railway systems. Traditional inspection methods, performed manually, are often slow, costly, and prone to human error, which motivates the adoption of computer vision and deep learning techniques. This work presents a comparative analysis of several deep learning models applied to the detection of rail anomalies using image data. The study investigates classical convolutional neural networks, hybrid approaches, visual transformers, and architectures specifically designed for railway inspection. It also includes preprocessing strategies and transfer learning techniques based on best practices from the literature. The expected outcomes include superior performance compared to traditional methods, improved robustness under varying imaging conditions, and the identification of the most suitable architectures for automatic defect detection. The overall goal is to support the development of accurate and accessible technologies for predictive maintenance of railway infrastructure.
\begin{center}
    \textbf{KEYWORDS}
\end{center}
Rail defect detection; Computer vision; Deep learning; Transfer learning; Vision Transformers; Railway infrastructure.
